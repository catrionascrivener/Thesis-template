

If you want to add references, you will need to create a .bib file. If you use Zotero, it is very easy to create this. Right click on the folder of references that you want to use, select `export collection', select `BibTex' from the drop down menu, and save the file somewhere on your machine. You can then upload the file into your navigation bar on the left. More details can be found in the Overleaf documentation.

% to cite a paper in the text (Adam et al., 2019) you use the following: \citep{adam}. The name might not be 'adam', but you will see a drop-down list appear, from which you can select the correct reference.
% to cite the year, when referring to Adam directly (2019) use the following: \citeyearpar{adam}
% to cite just the name Adam, without the year, use the following: \citet{adam}

An example of the uses of different referencing: a paper in \citeyear{scrivener_2019} found conflicting results to a previous paper \citep{scrivener_2019}, in which \citet{scrivener_2019} argue against the use of Word \citeyearpar{scrivener_2019}.